\chapter{Introduction}
%\begin{center}
%	\large{\textbf{Introduction}}
%	\vspace{0.3in}
%\end{center}
\pagenumbering{arabic}

% NEW PARAGRAPH AS REQUESTED
\paragraph{}
Unmanned Aerial Vehicles (UAVs), commonly known as drones, have emerged as powerful tools across domains such as surveillance, disaster management, agriculture, delivery logistics, and environmental monitoring. Their ability to operate autonomously or under remote control, access difficult terrains, and deploy quickly in dynamic situations makes them highly versatile.

\paragraph{}
In parallel, the demand for reliable wireless communication has grown rapidly, especially in remote, disaster-affected, or infrastructure-limited regions. Integrating UAVs with advanced communication technologies is therefore essential to ensure connectivity in such challenging environments.

\paragraph{}
LoRa (Long Range) is a low-power, wide-area communication technology designed for Internet of Things (IoT) applications. Operating in sub-GHz ISM bands, LoRa enables long-range communication with minimal energy consumption. It uses Chirp Spread Spectrum (CSS) modulation, which enhances resilience against noise and interference, making it suitable for smart cities, rural deployments, and large-scale sensor networks.

\paragraph{}
Combining UAVs with LoRa technology creates a flexible and scalable wireless communication platform. UAVs equipped with LoRa transceivers can serve as mobile gateways or relay nodes, significantly extending coverage and bypassing the limitations of fixed infrastructure. This capability is particularly valuable in emergency scenarios, where ground-based networks may be unavailable, and rapid communication setup is critical.

\paragraph{}
A typical UAV–LoRa architecture consists of distributed sensor nodes, LoRa transceiver modules, gateways mounted on UAVs, and centralized network/application servers. Data collected by the sensor nodes is transmitted via LoRa to the UAV gateways, which then relay the aggregated information to the server for processing. Designing such systems requires careful consideration of UAV payload weight, power consumption, antenna design, and communication reliability to maximize efficiency and flight endurance.

\paragraph{}
Despite its advantages, UAV–LoRa integration faces challenges such as limited bandwidth, UAV power constraints, signal interference, and spectrum regulations. Research efforts are addressing these issues through optimized flight path planning, adaptive deployment strategies, channel management, and energy-efficient communication protocols. Progress in lightweight hardware and battery technology continues to improve the feasibility of these systems.\cite{davoli2021hybrid}
