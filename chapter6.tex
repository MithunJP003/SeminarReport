





\chapter{Summary}

\paragraph{} 
The project on \textit{UAV-based LoRa Communication System} focuses on integrating unmanned aerial vehicles (UAVs) with low-power, long-range (LoRa) communication technology to create a reliable and energy-efficient communication network. Traditional ground-based LoRa networks are limited by terrain, obstacles, and line-of-sight issues, which reduce coverage and reliability. By deploying LoRa nodes on UAVs, these limitations are significantly reduced, enabling wider communication ranges and more reliable connectivity, especially in remote and challenging environments.

The background study highlighted the increasing role of UAVs in communication systems and the advantages of LoRa technology, including low power consumption, extended range, and suitability for Internet of Things (IoT) applications. Literature review further confirmed that UAV-assisted LoRa systems have been explored for disaster management, environmental monitoring, and smart agriculture, with promising results in enhancing network resilience and coverage.

The system architecture designed in this work integrates LoRa transceivers with UAV platforms. The design flow included UAV integration, LoRa link setup, data acquisition, and analysis of performance metrics such as Received Signal Strength Indicator (RSSI), Signal-to-Noise Ratio (SNR), and Packet Delivery Ratio (PDR). Experiments were conducted at varying UAV altitudes to study the impact on link quality and coverage distance.

Results demonstrated that UAV elevation significantly improves LoRa communication performance. At higher altitudes, signal propagation improved, extending coverage up to 2.5 km with near-perfect reliability, compared to 1.2 km for static ground nodes. Additionally, LoRa’s extremely low power consumption ensures suitability for UAV deployment without excessive energy drain. Performance comparisons confirmed that UAV-based LoRa outperforms conventional ground-only systems in terms of coverage, reliability, and energy efficiency.

In conclusion, UAV-based LoRa communication systems represent a practical and scalable solution for establishing temporary or emergency communication networks in scenarios such as natural disasters, remote monitoring, and defense applications. This study validates the feasibility and effectiveness of combining UAV mobility with LoRa’s long-range capabilities, paving the way for future research in autonomous deployment, swarm-based communication, and intelligent network optimization.

 
