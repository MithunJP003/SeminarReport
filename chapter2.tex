\chapter{Background and Literature Review}

\pagestyle{fancy}

\section*{Background}

Unmanned Aerial Vehicles (UAVs) have become one of the most rapidly advancing technologies in recent years, moving far beyond their early military use cases. Today, they are deployed in a wide variety of civilian and commercial domains such as precision agriculture, aerial mapping, infrastructure inspection, environmental monitoring, disaster response, and delivery logistics. Their popularity stems from their ability to operate autonomously or under remote control, to fly over terrains that are otherwise difficult to access, and to carry out time-sensitive missions with great flexibility. The constant development of lightweight sensors, advanced autopilot systems, and high-density batteries has further accelerated the adoption of UAVs across multiple industries.

At the same time, the Internet of Things (IoT) has emerged as a key technological driver, enabling interconnected devices and sensor networks to exchange data for automation, monitoring, and decision-making. A crucial requirement for IoT networks is reliable long-range wireless communication with minimal energy consumption. While technologies such as Wi-Fi, ZigBee, and Bluetooth offer short-range connectivity, they fall short when applied to wide-area sensor deployments. Cellular networks provide better coverage but are energy-intensive and not always available in remote or disaster-hit regions.

LoRa (Long Range) technology addresses these limitations by providing low-power, wide-area communication in sub-GHz ISM bands. Its use of Chirp Spread Spectrum (CSS) modulation makes it resistant to interference and effective in low signal-to-noise conditions. LoRa can achieve communication distances of several kilometers while consuming only a fraction of the energy compared to traditional wireless systems. Combined with the LoRaWAN networking protocol, it supports secure, large-scale deployments and adaptive data rates, making it an ideal choice for IoT applications in smart cities, agriculture, and rural areas.

The integration of UAVs with LoRa technology creates a novel and powerful communication paradigm. UAVs can act as mobile gateways, dynamically moving to locations where connectivity is needed most. Unlike fixed gateways, UAVs can be deployed quickly in areas where infrastructure is damaged or nonexistent. This makes UAV–LoRa systems highly suitable for disaster management, military operations, and temporary event-based monitoring. Moreover, UAVs can reposition themselves to optimize coverage, reduce shadowing effects from obstacles, and balance communication loads in a network.

Thus, the background of this seminar lies in the convergence of two rapidly growing fields—UAVs and LoRa communication. Together, they provide a scalable, energy-efficient, and adaptive solution to modern connectivity challenges.

\section*{Literature Review}

Research into UAV-based communication systems has expanded significantly over the past decade. Early investigations primarily explored the use of UAVs as aerial base stations for cellular and Wi-Fi networks. These studies demonstrated that UAVs could improve network coverage and user throughput in urban and rural scenarios. However, they also highlighted limitations such as the high energy consumption of cellular equipment, short coverage duration due to limited UAV battery capacity, and the complexity of spectrum licensing for cellular use.\cite{gallego2020enhancing}

As a result, recent research has shifted toward combining UAVs with low-power wide-area technologies such as LoRa. Experimental studies have shown that UAV-mounted LoRa gateways can extend network coverage far beyond what is achievable with ground-based gateways, especially in rural and mountainous regions. For example, field tests demonstrated that UAVs flying at moderate altitudes could achieve reliable LoRa communication links spanning 10–15 kilometers, even in areas with obstacles that degrade ground-based coverage.

Several works have investigated optimal UAV deployment strategies for LoRa networks. Flight altitude, trajectory, and hover duration play critical roles in maximizing coverage and ensuring energy efficiency. Simulation-based studies suggest that dynamic flight path planning—where UAVs adjust their altitude and position based on node distribution and environmental factors—can significantly improve network performance. Other research has proposed swarm-based deployments, where multiple UAVs cooperate to provide scalable and redundant coverage for large areas.\cite{ghazali2021systematic}

Case studies in disaster recovery highlight the effectiveness of UAV–LoRa integration. In scenarios where earthquakes, floods, or hurricanes damage terrestrial communication infrastructure, UAVs equipped with LoRa gateways can establish a temporary communication backbone within hours. This allows rescue teams to collect sensor data, track personnel, and coordinate operations more effectively. Similar research has also explored agricultural monitoring, where UAV–LoRa systems are used to gather data from distributed soil, crop, and weather sensors across wide farmlands.

Despite these promising results, the literature also identifies key challenges. Bandwidth limitations restrict LoRa to low-data-rate applications, which may not suffice for high-bandwidth sensor types such as video or imaging payloads. UAV power constraints remain another critical issue, as extended flight times are necessary to maintain communication coverage. Additionally, regulatory and spectrum management concerns pose obstacles to large-scale deployments in different countries.

Overall, the body of literature indicates that UAV–LoRa systems offer a cost-effective, resilient, and energy-efficient solution for next-generation wireless communication. At the same time, ongoing research is focusing on solving practical challenges such as energy management, adaptive spectrum usage, and intelligent multi-UAV coordination.